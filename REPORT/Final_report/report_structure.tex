
Evaluation criteria:
\begin{list}{-}{ }
	\item Project Management | current content is good but more is required than in the progress report
	\item Technical approach | clearer requirements, discuss evaluation strategies??
	\item Testing and Evaluation | talk about unit testing, integration testing (which was limited to interface generation and system penetration testing done using the tool) 
	\item Achievement | 
	\item Main report | literature review is as needed, good writing style, waiting for email response
	\item Knowledge and understanding (+ Presentation)
\end{list}


Report structure:
\begin{enumerate}
	\item Title page
	
	\item Abstract
	
	\item Acknowledgement to Julian
	
	\item Statement of originality
	
	\item Auto generated content
	
	\item Introduction | what is the situation in this area currently, what is the problem and what approaches I would suggest to solve it (increase awareness, user education, making pen testing easier for people )
	\begin{enumerate}
	  	\item Review progress report
	  	\item little changes expected
	\end{enumerate}
	
	\item Background research
	\begin{enumerate}
		\item Review progress report (Many changes expected)
		\item Threat Modelling in general:
		\begin{enumerate}
			\item talk about Microsoft model
			\item talk about book threat model
			\item talk about alternative threat modelling approaches
		\end{enumerate}
	\end{enumerate}
	
	\item Project specifications and Analysis
	\begin{enumerate}
		\item Goal
		\item Requirements
		\begin{enumerate}
			\item Functional
			\item Non-Functional
		\end{enumerate}
		\item Scope
		\item Risk analysis
		\item Existing alternatives
		\begin{enumerate}
			\item Possible problem approaches
			\item Chosen Approach
		\end{enumerate}
	\end{enumerate}

	\item Design
	\begin{enumerate}
		\item Design patterns used (M-V-C)
		\item Technical specification
		\begin{enumerate}
			\item Platform choice - desktop app
			\item Language choice
			\item Libraries choice
			\item Development tools choice
		\end{enumerate}
	\item Source control
	\item Class diagrams and planning
	
	\end{enumerate}

	\Implementation details
	
	\item User preferences and setting files
	\item Interface loader and yml parsing
	\item Interface generation, recursion and extendability
	\item Multithreaded run and options to stop execution
	\item Dynamically bind seperate output for each interface - output abstraction using functional programming
	\item Interface tabular structure in accordance with methodology tool separation
	
	\item Threat Model design using tabular structure 
	\item Dynamic and responsive information display kept up to date in multiple places using QT signal-slot pattern
	\item Global scope Threat Model item cache (remembers entered values even between separate Threat Models to easy repetitive assets imports)
	\item Threat sorting and information add-edit-delete capabilities
	\item Interface flag and value suggestion generation in extendable manner parsing only current threat information (done in native style as text completion)
	
	\item Threat Model persistence via pickling
	\item Expected tool interface for new/open/save and save as with keyboard shortcuts and, obviously, remembering saved file existence, remembering saved/unsaved state and verification before discarding unsaved information
	\item Inter version compatibility due to pickling (can load not exactly the same attribute files and fill in not existing attributes with defaults)
	\item export to tabulated json as a universal format
	\item Global cache clearance separated into assets/technologies and entry points
  
\end{enumerate}

