
Evaluation criteria:
\begin{list}{-}
	\item Project Management | current content is good but more is required than in the progress report
	\item Technical approach | clearer requirements, discuss evaluation strategies??
	\item Testing and Evaluation | talk about unit testing, integration testing (which was limited to interface generation and system penetration testing done using the tool) 
	\item Achievement | 
	\item Main report | literature review is as needed, good writing style, add about pen testing in general
	\item Knowledge and understanding (+ Presentation)
\end{list}


Report structure:
\begin{enumerate}
	\item Title page
	
	\item Abstract
	
	\item Acknowledgement to Julian
	
	\item Statement of originality
	
	\item Auto generated content
	
	\item Introduction 
	\begin{enumerate}
		\item IoT tech adaptation (current situation)
		\item security flaws (problems)
		\item project goals (the solution)
	\end{enumerate}
	
	\item Background research
	\begin{enumerate}
		\item Review progress report (Many changes expected)
		\item Threat Modelling in general:
		\begin{enumerate}
			\item talk about Microsoft model
			\item talk about book threat model
			\item talk about alternative threat modelling approaches
			\item DREAD rating
		\end{enumerate}
	\end{enumerate}
	
	\item Project specifications and Analysis
	\begin{enumerate}
		\item Requirements
		\begin{enumerate}
			\item Functional
			\item Non-Functional
		\end{enumerate}
		\item Scope (limitations)
		\item Risk analysis
		\item Existing alternatives
		\begin{enumerate}
			\item Possible problem approaches
			\item Chosen Approach
			(Platform choice - desktop app)
		\end{enumerate}
	\end{enumerate}

	\item Design
	\begin{enumerate}
		\item Class diagrams and planning
			\item Tools and Techniques
			\begin{enumerate}
				\item Language choice
				\item Libraries choice (QT - native gui, means that shortcuts work as well)
				\item Development environment
				\item Source control
				\item Techniques
				\begin{enumerate}
					\item Design patterns used (mvc, observer(signal-slot), add some more)
				\end{enumerate}
			\end{enumerate}
		\item Application structure (discuss parts of the class diagram)
		\begin{enumerate}
			\item main - IntefaceLoader - configManager - (other high level interfaces) connection
			\item InterfaceLoader - Interface - InterfaceGui connection
			\item InterfaceGui generation
			\item Manager - Worker - job - response pattern
			\item ThreatModel - Asset - Technoliogy - EntryPoint - Threat structure
			\item Each tab Gui - each tab Gui Controller - ThreatModelGuiController - ThreatModel
			
			\item importing new tools to the toolset as easy as creating a yml file (is not indended to be used by unexperienced users)
			\item toolset categorization
		\end{enumerate}
			
	\end{enumerate}

	\item Implementation details
	\begin{enumerate}
		\item why yaml?
		\item importing new tools to the toolset as easy as creating a yml file (is not indended to be used by inexperienced users)
		\item toolset categorization
		
		\item third part code used
		
		\item earlier designs and prototyping
		
		\item User preferences and setting files
		\item Interface loader and yml parsing
		\item Interface config tool structure extendability
		\item Functional programming to seperate output of each interface - output abstraction using functional programming
		\item Interface tabular structure in accordance with methodology tool separation
		
		\item Threat Model design using tabular structure 
		\item Global scope Threat Model item cache (remembers entered values even between separate Threat Models to easy repetitive assets imports)
		\item Threat rating according to DREAD and every part add-edit-delete capabilities
		\item Interface flag and value suggestion generation in extendable manner parsing only current threat information (done in native style as text completion)
		
		\item Expected tool interface for new/open/save and save as with keyboard shortcuts and, obviously, remembering saved file existence, remembrance of saved/unsaved state and verification before discarding unsaved information
		\item export to tabulated json as a universal format
		\item Global cache clearance separated into assets/technologies and entry points
		
		\item Toolset provided - talk about each of the tools in the toolset
		\begin{enumerate}
			\item apktool
			\item binwalk
			\item dirb
			\item hcitool
			\item hydra
			\item nmap
			\item sdptool
			\item sqlmap
			\item wfuzz
		\end{enumerate}
		
		\item scalability (not very good, it is expected that people don't use too many tools as it wouldn't be convenient for them)
	\end{enumerate}

	\item Testing
	\begin{enumerate}
		\item unit testing
		\item integration testing (only for interface gui generation)
		\item testing usefulness by pen testing IoT system
	\end{enumerate}

	\item Project management
	\begin{enumerate}
		\item prototyping stages
		\item Test driven development
		\item Agile planning
		\item MoSCoW prioriterization
		\item Sprints scheduled as supervisor meetings
		\item Commit history
		\item rapid development
		\item Gannt charts (planned? actual?)
		\item Project management evaluation
		\begin{enumerate}
			\item Deviation from plan
			\item Rushing
			\item Cutting corners
			\item Time sinks
		\end{enumerate}
	\end{enumerate}

	\item Evaluation
	\begin{enumerate}
		\item Did it solve the problem?
		\item Was the approach appropriate?
		\item Provided functionality - mention project brief and it's goals
		\item Usability
		\item Extendability
		\item Future work
		\item Final sum-up
	\end{enumerate}

	\item References
	
	\item Appendices
	\begin{enumerate}
		\item Project Brief
		\item Unit testing results
		\item Pentesting screenshots
		\item Gannt charts
		\item Risk analysis
		\item Content of Design Archive
	\end{enumerate}
  
\end{enumerate}

