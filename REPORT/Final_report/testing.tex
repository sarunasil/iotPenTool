\section{Testing}\label{testing}
Project testing was performed in two ways: software testing during Test-Driven (TD) development and a tool use case analysis. 

\subsection{Software testing}
	As it has been mentioned before, application testing process has been integrated into software development by following TD development method. Full test log can be found in appendix \ref{sec:appendix-software-tests}. There is a total of 109 separate test cases being tested.

	\subsubsection{Unit testing and Integration testing}
	For each individual piece of software functionality an appropriate unit test had been created before starting implementation. Then a function passing that test was written. Afterwards, more test cases were added to test alterations in control flow, finally a code was refactored to comply with all test cases.
	
	Following this workflow, appropriate pieces of code were tested during their development. It also matches with individual Sprints; therefore, fully tested pieces of software were presented during each after-Sprint supervisor meeting.
	
	Software test log in Appendix \ref{sec:appendix-software-tests} is a mixture of Unit and Integration tests. Due to the TD process, their distinction is not obvious as some cases evaluate a combination of object functionalities (e.g. parsing and creation of interface files) and others only specific functions (e.g. adding a new Asset object to ThreatModel).


	\subsubsection{GUI testing}
	GUI prototypes were tested using manual testing process during development. Program GUI was not a priority for this project and the testing that has been completed for it was not as in-depth as for the back-end software part. Program GUI functionality has been checked again during the Usability testing phase.


\subsection{Usability testing}
	Usability testing demonstrates prototype usefulness in close to real world system setup. The goal of it is to go through the whole pen testing process using the tool as an end user would. This section describes the tested IoT system, concluded penetration testing results and evaluates how practical the tool is.
	
	The IoT system chosen for testing is a home alert system built for a different module. It contains a variety of distinct network solutions and technologies (e.g. Bluetooth LE, WiFi, cloud and mail service); thus, simulating a large real life IoT network. For this case a "grey box" testing approach had to be employed. Ideally, penetration testing using the "black box" approach would have been preferred as then real first-time system analysis using the IoT penetration tool prototype would have been achieved. Regrettably, a large part of the alert system was developed by the author of this report, therefore, some system insight knowledge could not be denied. 
	
	The penetration testing started based on the implemented IoT penetration testing methodology, followed by appropriate tests and threat model updates. Screenshots of this system penetration testing process can be found in the Appendix \ref{sec:appendix-pen-testing} and the saved file is included in the submission. 
	Distinction between system assets, technologies and entry points as well as the connections between these entities helped to identify the weaknesses of the system. Then Nmap, dirb, wfuzz, sdp, hcitools and hydra tools were used respectively to assess the potential threats. Neither firmware, nor mobile application analysis took place as they were not needed in this case.
	The alert system pen test has shown several significant security issues mostly due to the use of HTTP instead of HTTPS protocol and weak to non-authentication. Physical system component vulnerabilities could also be exploited to disrupt the workflow. 
	
	The usability test has shown that the prototype can be used to test real life IoT systems. The GUI is intuitive, responsive and functional, although, not well-styled. The penetration tools included run as expected, however, it is advisable to use the prototype as a root user because some parts of the tools that are included, require to be run with elevated privileges. The interactive keywords transfer from threat model to respective pen tool fields works as intended. Overall, the IoT pen testing tool works well and is suitable for the task.
