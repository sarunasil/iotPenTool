\section{Evaluation}
	
	\subsection{Requirements evaluation}
	The prototype provided complies with all but one previously defined functional and all but one non-functional requirement. 
	
	Importing or removing penetration tools from the tool set is uncomplicated by moving interface files from the dedicated folder. The proof-of-concept tool set contains various tools including the ones for LAN network scanning and Bluetooth communication protocols. The implemented threat model is designed to be updated with new data from the penetration test runs, as well as mapping appropriate keywords from the threat model to the individual penetration tools. After testing is completed, the threat model together with the included testing results can be exported as structured; JSON output to be used later or displayed in a separate report. 
	The functional requirement which was not met is the tool output chaining. This specification had to be refused due to its complicated nature involving universally supporting parsing of penetration tools output and its limited cases of use.
	
	The tool has been tested and run on Ubuntu OS using Python and Pip package manager which guarantees its compatibility with other Linux operating systems that support the required dependencies. It also implements multithreading, thus, moving all the computationally intense tasks away from the main program threat and ensuring quick program response. The prototype does not scan for devices on the network itself but employs well-known pen tools to complete these tasks, it shifts the responsibility to Nmap and Hcitool implementations which can perform scans on huge networks without any difficulties. 
	The unfulfilled, non-functional requirement was the complete encapsulation of the tool in one portable program. Due to a shorter development period and the relatively low additional value of this feature, the requirement was not completed. The prototype packaging is somewhat finished as third-party dependencies can be installed automatically using a package manager.
	
	The final result is in accordance to the project Brief included in Appendix \ref{sec:appendix-preject-brief} and the project goals raised at the start of development process are met. A slight shift from the project brief goals is that the final prototype focuses on the use of IoT penetration methodology and not just penetration testing.
	
	\subsection{Usability and testing}
	The tool usability has been evaluated during an actual IoT alert system penetration testing. The system was tested using "grey box" method which combines internal system knowledge with the outside attacker methods. The IoT penetration tool proved to be usable and suitable for this kind of penetration testing, nevertheless, additional work in tools GUI design would be welcomed.
	
	The tool has been tested with a combination of unit and integration tests with an emphasis on the tools back-end functionality. The front-end and GUI testing has been done manually and by doing the evaluation of the alert systems.
	
	\subsection{Future work}\label{future-work}
	A few parts of the IoT penetration tool could be improved in the future. \newline
	At the current moment, the system can pipe penetration tools output to any destination, however, only the most basic terminal-like display is included. A more user-friendly way of presenting penetration tools output would greatly improve the usability of the tool. During the project development it was decided to implement a generic approach as individual penetration tools use their own output formatting and may support other forms of data representation. As the tool is aimed at supporting any command-line based penetration tool, a generic approach which can be extended to other formats, makes more sense. 
	
	For the prototype to include advanced tool functionality GUI design has been neglected and a properly formatted GUI could make the tool more readable. As the tool GUI design is not optimized some options or fields are not displayed in the most efficient manner. The same could be said for the colour scheme and text font styles.
	
	Effort could potentially be made to expand the current tool set by writing interfaces for a package sniffer or for penetration tools used in other areas e.g. firmware analysis. As the project requirement was to provide an extendable penetration tool platform and a proof-of-concept tool set, only the basic cases of use are covered. 
	
	Some additional work could also be done for the tool to convert structured threat model format into a user-friendly interactive report displaying key penetration testing results. The prototype provides structured data which could be represented in any way and in any format needed. Report generation in a particular form would limit the gathered information to a single format that could not be easily reused. 
	
	\subsection{Conclusion}
	This project addresses the rise of IoT related cyber security risks and their exploitation. It provides a platform where threat modelling is combined with penetration tools. The tool is trying to fill the void of dedicated IoT penetration testing solutions which could be used by people with little cyber security related experience. It is essentially a highly extendable IoT penetration testing platform closely linked with IoT threat modelling methodology. The two are coupled in order to guide the user to properly structure and document the penetration testing process. The proposed "IoT Penetration Testing Toolset" could also be called a penetration testing assistant as it helps the user to structure the system evaluation process. It is hoped that the tool would assist in creation of more secure consumer and industry level IoT systems.
