\section{Random stuff}

The most popular and the most convenient method of interaction with an IoT device is via wireless network. There are many different communication technologies developed for this purpose. All of them were designed to accomplish specific tasks, therefore they vary in range, connection speed, energy consumption and security. Some of the most widely used are WiFi, Bluetooth, ZigBee, GSM and Z-Wave \cite{cookbook}. To devise a proof of concept, two different protocols are sufficient, therefore this project will concentrate on WiFi and Bluetooth communication. WiFi is based on IEEE 802.11 standards, it is no different to LAN communication. The two most ways of sending information is UDP and TCP protocols which act very differently \cite{4359944}. By itself WiFi communication is not encrypted, therefore if application developers did not secure ongoing traffic, such data is vulnerable to network sniffing attacks. Recent versions of Bluetooth, contrarily have built-in encryption, therefore sniffing it is rather complicated \cite{scarfone2012guide}. Nevertheless, Bluetooth devices are still vulnerable when they connect to an end-point for the first time (pairing) \cite{scarfone2012guide}. It is easier to exploit specific protocol vulnerabilities with tools already designed for that purpose. As the tool set is designed to take re-use best parts of already trusted penetration tools, it will be able to retain generality still succeeding in fine-grain testing.



This tool was created to solve the problem of lack of clarity and usability of penetration testing in a IoT technology environment. It seems that penetration testing if not the integration testing in general is a rather side topic and not a definite must for IoT developing companies. Therefore, this project tries to solve this problem by providing an little skill requiring toolset that allows users to employ tested and trusted penetration tools to do penetration testing themselves by following instructions. That is a clear advantage and narrows the gap between developers and security experts in hopes of advertising penetration testing techniques and providing means for developers to test their own software.


Evaluation criteria:
\begin{list}{-}
	\item Project Management | current content is good but more is required than in the progress report
	\item Technical approach | clearer requirements, discuss evaluation strategies??
	\item Testing and Evaluation | talk about unit testing, integration testing (which was limited to interface generation and system penetration testing done using the tool) 
	\item Achievement | 
	\item Main report | literature review is as needed, good writing style, add about pen testing in general
	\item Knowledge and understanding (+ Presentation)
\end{list}
	