\section{Random stuff}

This tool was created to solve the problem of lack of clarity and usability of penetration testing in a IoT technology environment. It seems that penetration testing if not the integration testing in general is a rather side topic and not a definite must for IoT developing companies. Therefore, this project tries to solve this problem by providing an little skill requiring toolset that allows users to employ tested and trusted penetration tools to do penetration testing themselves by following instructions. That is a clear advantage and narrows the gap between developers and security experts in hopes of advertising penetration testing techniques and providing means for developers to test their own software.


Evaluation criteria:
\begin{list}{-}
	\item Project Management | current content is good but more is required than in the progress report
	\item Technical approach | clearer requirements, discuss evaluation strategies??
	\item Testing and Evaluation | talk about unit testing, integration testing (which was limited to interface generation and system penetration testing done using the tool) 
	\item Achievement | 
	\item Main report | literature review is as needed, good writing style, add about pen testing in general
	\item Knowledge and understanding (+ Presentation)
\end{list}
	
	
The testing started by using nmap to scan local network expecting to discover the "system hub" and the networks router. After a simple port scan some open ports were found on the network router (unrelated to the alert system but nevertheless a security concern) as well as SSH port and port 8080 open on the system hub. System hub uses the port 8080 to listen to configuration API requests. As the system hub was deployed on a Raspberry Pi which also had an unrelated web server running on it, thus those results will be ignored in this report. The SSH login was tested using Hydra tool and the 8080 port traffic could be potentially sniffed to see the device communication. While port scanning the cloud server an SSH login was also found on the cloud server the system hub connects to, thus it can be tested with Hydra. System hub web interface was tested using dirb and wfuzz to look for open end-points but a more efficient approach would have been to try sniffing system owners traffic. For all user interaction the alert system uses HTTP unencrypted traffic which is a highly dangerous.
The alert system hub communicates with the sensor and ringer devices using Bluetooth Low Energy (BLE) protocol. sdp and hcitool were used to discover related BLE devices, regrettably, it was not possible to establish connection to them. The threat model addresses few vulnerabilities related to capturing and jamming BLE traffic in order to disrupt communication and no authentication during the devices pairing process.
The system in question does not have a web page interface nor a mobile app related to it, thus it could not be tested for XSS or similar vulnerabilities. Using internal knowledge of the system, it was noted that the system hub uses a Mongo database to store configuration files, but it could not be accessed from outside the Pi. For the same reason the system is not vulnerable to SQL injections. 
	
	

This report is divided into sections starting from background research which covers penetration testing, most common IoT technologies and then shifts to explain IoT penetration testing threat modelling approaches and industry established patterns. Afterwards, project requirements and possible alternative approaches are discussed. Then the report considers numerous design and development decisions that were taken during the course of project progression. Application architecture and implementation details are then described in the following sections. Application design and development is supported by extensive software testing and an IoT system analysis. Subsequent sections cover project management assessment, final deliverable evaluation and possible future work for further framework development.