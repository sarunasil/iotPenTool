

\section{Implementation}
This section of the report covers project prototype details more in-depth emphasising on implementation decisions taken during development. It also covers the list of well-known penetration tools included in the tool set.

\subsection{Development process}
\subsubsection{Development}
\subsubsection{Debugging}
\subsubsection{Deployment options}


\subsection{Tool included in the tool set}
	
	\begin{enumerate}
		\item apktool
		\item binwalk
		\item dirb
		\item hcitool
		\item hydra
		\item nmap
		\item sdptool
		\item sqlmap
		\item wfuzz
	\end{enumerate}

\subsection{Third-party code used}

\subsection{Implementation details}

	\subsubsection{Application preferences file}
	
	
	\subsubsection{Tool configuration file structure}
	why yaml? How are yaml files structured
	importing new tools to the toolset as easy as creating a yml file (is not indended to be used by inexperienced users)


	\subsubsection{Tabular design}
	Threat Model design using tabular structure 
	Interface tabular structure in accordance with methodology tool separation
	
	Toolset categorization in tabs 
	
	
	\subsubsection{Graphical User Interface functionality}
	Convenient add-edit-delete capabilities
	Item caching and cache clearing
	Open-Edit-Save threat models and use of native shortcuts, obviously, remembering saved file existence, remembrance of saved/unsaved state and verification before discarding unsaved information
	
	
	\subsubsection{Scalability}
	(not very good, it is expected that people don't use too many tools as it wouldn't be convenient for them)
	
	Great extendability! Separated output using functional programming


	\subsubsection{Exporting and cross-compatibility}
	 export to tabulated json as a universal format

