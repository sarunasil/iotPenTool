\section{Project management} \label{project-man}
This section covers time and workload management during the whole project development period. It explains important project planning decisions, covers the strategies used for time planning and work distribution. It also compares the project schedule from the Progress report with the final project schedule. Finally, it evaluates project management and explains the deviations from the initial plan.

\subsection{Early prototyping period}
	At the end of the first semester exam period, it has become clear that before proper development could be started, some degree of experimentation is required. Firstly, experiments to get familiar with Qt development patterns, functionality and convenience software were needed. Then, research into data structure parsing, finally settling with yaml data format. Thirdly, some time was needed to test command line string generation and execution in the terminal from Python.
	
	Unfortunately, there was a need for more background research especially in the specific penetration tools that would be most suitable to perform the test for IoT devices. The delays shifted the project schedule a few weeks and postponed the actual prototype creation. 
	
	\subsection{Agile methodology}
		A range of Agile practices were followed during the whole project timespan. Those include well-defined sprints, demos, task prioritization, Test-Driven and rapid development. As this project was done by only one person, not all Agile practices even make sense as there was no cooperation between team members involved. In most cases, Agile practices were seamlessly adapted in order to efficiently produce robust software and keep track of the remaining and completed work.
	
	
		\subsubsection{Sprint planning}
		Sprint task planning took place before each sprint and sometimes had to be altered during the sprint week. The tasks were compiled in a feature-orientated manner, dedicating whole sprints to individual parts of the application. This approach minimized the need for code refactoring in later sprints. Sprint plans were stored and updated using source control, removing the need for dedicated sprint planning programs as there was little to be gained from it in a one-person developer’s team. 
		
		Each sprint would end by a supervisor meeting which conveniently was similar to Agile sprint demo practice. During each meeting sprint progress and completed features were discussed as well as defining tasks for the upcoming iteration. Similarly to a customer’s demo feature improvements and suggestions were considered and the sprint plan adjusted accordingly.
		
		Appendix \ref{sec:appendix-sprint-plans} includes formatted sprint plan according to MoSCoW prioritization. Reader may notice that different sprints lasted a varied number of days, this was needed to accompany other assignments and varied workload. Some tasks (marked red) were not completed on time or moved to the next sprint.
	
		\subsubsection{MoSCoW prioriterization}
		Using MoSCoW requirement prioriterization technique each Sprint task was ordered according to the necessity and potential product value. Every sprint task was categorised into four groups: Must, Should, Could, Won't (highly unlikely) on the day of the supervisor demo meeting. This structured approach kept the author's focus on the most important features assigned for that Sprint. A simple list of prioritized tasks provided an uncomplicated way of following sprint progress and the remaining work.  
		
		\subsubsection{Github source control}
		Github source control platform has been used frequently to log project progress and development. Following rapid programming principles, it was aimed at frequent and single functionality commits to aid debugging process. Git managed notes also served as a sprint log between development cycles. Conveniently, Github also generates graphs illustrating project development. 
		
		Project statistics can be found in Appendix \ref{sec:appendix-github}. Commit history graph and the code frequency graph shows rapid development period of five sprints from the end of February till the beginning of April and the early experimentation phase before it. Commits in the last weeks of April are due to the project report being written.
		
		
	\subsection{Test driven development}
		From the start of the project Test-Driven development (TDD) technique has been chosen as the most time and resource efficient way of delivering robust code. Due to the delay suffered from the early prototyping phase, development schedule had to be pushed into a narrow window with little to no time for other setbacks. The code had to be developed efficiently, according to Sprint plan and fully tested. 
		
		As each Sprint was finalized by a supervisor meeting during which working code had to be provided, a high risk of (un)intentional cutting of corners was possible. TDD approach was ideal for such setup in order to ensure proper software testing and delivery of working code.
		
	\subsection{Project schedule}
		Gannt charts illustrating the expected and the actual project schedules are included in the Appendix \ref{sec:appendix-gantt}. As it is apparent, due to miscalculation and confusion between report submission and the viva dates, the actual project development had to be shortened by a few weeks; that resulted in moving the report writing into earlier weeks. The need for additional background research and longer prototyping periods forced the development sprints to be pushed back a few weeks and contracted into a smaller time window. Nevertheless, the expected and the actual schedules do not differ significantly as most of the planned activities were completed in the predicted time slots.
		
	\subsection{Risk management}
		The project risk table is included in the Appendix \ref{sec:appendix-risk}. The table contains the possible risks and their mitigation techniques. It can be concluded, that there were no unexpected risks during the project development stages. Prototyping and research phases took longer than expected first, however, the situation was resolved via the applied mitigation techniques.
		
	\subsection{Project management evaluation}
		Overall, the project time and resource management were backed up by the methods and techniques used in industry following Agile development framework. Each of the project phases had been allocated specifically loose time frames leaving enough flexibility for alterations. In general, the defined schedule was met with minimal changes. 
		
		Prototype coding tasks were divided into reasonable sprints and prioritized using MoSCoW technique. Sprint deadlines were flexible depending on sprint task complexity. Task prioritization helped with the project progress tracking and further scheduling. Development and testing process integration ensure that the code is robust and self-encapsulated.
		
		Project risk table and contained mitigation techniques allowed for straightforward situation adjustments minimizing the time loss. Schedule alterations only had to be made for longer background research and prototyping periods.
		
		\subsubsection{Cutting corners}\label{sec:cut-corners}
		In order to comply with compressed programming schedule, some elements had to be omitted. In particular, prototype GUI element style and design were not taken into consideration. Additionally, a requirement for chaining individual penetration tool execution had to be skipped (further detail in the Evaluation section). 
		
		Although prototype is self-explanatory and intuitive, there is no documentation provided as it had to be dropped out due to time constraints. The prototype deployment and compatibility with other systems is also not fully tested, despite the fact that there are no reasons for it to be incompatible. 
		
		\subsubsection{Time sinks}
		Not surprisingly, project functionality requiring the most time to implement, was the GUI interface and its testing; especially the threat model GUI. Prototype tool set assembly and configuration took more time than expected which had to be moved to the next sprint. 
		
		Additionally, it has not been considered how long data visualization for the project report will take. Particularly, a big number of diagrams was needed through the report to thoroughly explain particular prototype design or implementation details. 
