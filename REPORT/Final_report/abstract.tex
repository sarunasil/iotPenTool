


\section{Abstract}

Internet of Things (IoT) devices can be incredibly useful for everyday activities and are used for a wide range of applications. Unfortunately, securing these devices is not always the manufacturers top priority. This introduces a new, numerous and questionably protected subset of Internet connected assets. Penetration testing can be used to point out security faults and vulnerabilities that are most likely to be exploited by a malicious entities. 

This project aims to simplify IoT system penetration testing (pen testing) providing clear guidelines as well as an interactive penetration testing assistance tool prototype aimed for people with little technical knowledge. It is achieved by combining well-known threat modeling method with existing pen tools and emphasizing the gradual system model buildup and re-iteration process. The application threat model dynamically binds discovered system resources exposing their correlations and generating suggestions for future tests.

Apart from the interactive threat model this project utilizes a set of well-known command line based security tools and offers an auto-generated customizable graphical-user interface to simplify their usage. The key piece of the tool functionality is its extend-ability and versatility allowing new penetration tools to be easily included depending on user needs.

Report firstly explains the reasons for IoT security risk number increase and suggest penetration testing as a method of mitigating them. It then documents the design, development and testing phases for a prototype of IoT penetration testing assistance tool. Finally, it provides project evaluation and suggests future work.