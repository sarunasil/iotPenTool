\section{Introduction}

\subsection{IoT technology adaptation}
Modern society has seamlessly adapted to using gadgets and appliances that 30 years ago only existed in futuristic science fiction movies. Smart devices have slowly become a part of daily lives. Self-driving trucks are no more something out of a distant future\cite{freedman_2017} and the Internet connected appliances such as light bulbs and smart coffee makers are already finding their place in common people’s homes. It was estimated that in 2017 there were approximately 27 billion connected IoT devices around the globe. This number is expected to increase by approximately 12 \% each year until it reaches 125 billion by 2030 \cite{ihs-markit}. For comparison, in 2014 there were around 1.57 billion smart phone owners worldwide, this number is bound to reach 2.87 billion by 2020 \cite{statista}. Despite the increasing number of smart phones, they are limited by the number of people using them. This is not the case with smart Internet of Things devices. For example, an industrial factory may contain hundreds of different sensors ranging from management department AC connected thermometers to ensure staff comfort and electricity consumption meters to pressurized explosive gas cylinder sensors that are vital to ensure safety and efficiency of the factory. The IoT networks and stand-alone "Smart things" technologies are booming \cite{gartner2018} and will only increase in numbers. Unsurprisingly, such growth leads to privacy, security and regulation concerns as it is hard to control. Although these technologies have greatly complimented people’s lives, one cannot help wondering if their owners can manage these numbers of Internet connected devices properly.

\subsection{Security flaws}

Even though IoT devices collect incredible amount of users’ sensitive information and it is starting to cause processing and security concerns \cite{7069995} little is being done to ensure proper security of these devices. As it has been shown by the Mirai botnet attack in 2016, IoT technologies can be used to create record breaking botnets and disrupt more traditional Internet services\cite{203628}. If proper measures are not taken it is likely that larger scale IoT network exploitation and malicious activities will take place. As manufacturers are being pushed by the market growth to develop new products faster to keep up with their competitors, proper security testing is becoming a sort of luxury that usually not everyone can afford. Moreover, it appears that consumers are not concerned about their device security\cite{iotm}. As studies have shown even Internet connected vehicles can be hacked into and controlled remotely\cite{8071577}. As Internet connected devices are exposed to many more threats than stand-alone electronic devices, steps must be taken to ensure that IoT gadgets and Internet connected devices are resilient and safe to use.

Due to vast differences in deployment environment as well as technologies in use, proper IoT testing is hard to ensure. One reason that severely complicates the testing process is the tendency of manufacturers to use third-party technologies and services during the development stages. A single IoT device firmware may be written by a mix of contracted developers known as Original Design Manufacturers (ODM) and in-house developers hired by hardware manufacturer - Original Equipment Manufacturer (OEM), then the application code itself may be supplied by a completely different company\cite{cookbook}. The problem occurs when OEM and ODM reaches the stage where their code bases must be merged. The ODM may provide only the binary files or an SDK for the OEM. Therefore, if that happens Original Equipment Manufacturer (OEM) which is responsible for distributing firmware, managing it and releasing updates, does not have full access to the code. Moreover, IoT networks comprise a big number of various devices that are responsible for diverse functions. Each device individually may be manufactured by a different supplier and their OEMs accordingly. As no individual link of the supply chain possesses access to the full infrastructure source codes, it may be impossible to thoroughly test the system as a whole. It becomes apparent that the black box type penetration testing and vulnerability scans may be the only acceptable strategy aiming to secure complicated IoT systems.

As the field in question is rather new, diverse and rapidly developing, it is hard to find knowledgeable specialists and tools developed specifically for IoT penetration testing. There are numerous blogs and projects offering a various level of detailed guides to IoT penetration testing\cite{github}. Some organizations including IEEE and ETSI have released technology-specific standards as well as security guidelines but none of them have tried to cover IoT in general\cite{Zhao:2013:SIT:2584913.2585964}. 

\subsection{Project goals}
The purpose of this project is to simplify IoT penetration testing and provide a framework for future development. The project summarizes general IoT cybersecurity assessment concepts and presents them in a user-friendly manner. Moreover, this project aims to address the lack of dedicated IoT penetration testing software and provides an extensive framework prototype whose purpose is to reuse the existing penetration tools to test IoT systems. The tool is designed for users from software development background but with little knowledge of cybersecurity or pen testing. 

As the IoT technology stack is so diverse it is impossible to design a tool that covers all the cases of its use. Instead, this project aims to provide a "tool box" of well-known and proven penetration tools and has the function to conveniently add new tools in case of need. The application is aimed to be highly customizable and coherent.

Finally, in order to simplify the re-use of the proposed threat model and offer a practical solution, the developed penetration testing tool prototype must be closely bound with the IoT threat model concepts. A way of conveniently linking the information stored in the threat model and the "tool box" tools would greatly increase the prototype usability and justify its development.

