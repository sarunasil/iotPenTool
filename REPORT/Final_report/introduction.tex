\section{Introduction}

Modern society seamlessly adapted to using gadgets and appliances that 30 year ago only existed in futuristic science fiction movies. Smart devices have slowly become part of everyday lives. Self driving trucks are not something out of a distant future anymore\cite{freedman_2017} and Internet connected appliances such as light bulbs and  smart coffee makers are finding their places in common people homes. It was estimated that in 2017 there were around 27 billion connected IoT devices around the globe. This number is expected to increase by approximately 12 \% each year until it reaches 125 billion by 2030 \cite{ihs-markit}. For comparison in 2014 there were around 1.57 billion smart phone owners worldwide, this number is bound to reach 2.87 billion by 2020 \cite{statista}. Although, smart phone numbers are growing fast, they are limited by the number of people using them. This is not the case with smart Internet of Things devices. For example an industrial factory may contain hundreds of different sensors ranging from management department AC connected thermometers for staff comfort and electricity consumption meters to pressurized explosive gas cylinder sensors vital to ensure factories safety and efficiency. The IoT networks and stand-alone "Smart things" technologies are booming \cite{gartner2018} and will only increase in numbers. Unsurprisingly, such growth is starting to cause privacy, security and regulation concerns as it is hard to control. Even though these technologies have greatly complimented people lives one can not stop wonder can their owners properly manage these numbers of Internet connected devices.

Even though IoT devices collect incredible amount of users sensitive information and it is starting to cause processing and security concerns \cite{7069995} little is being done to properly secure these devices. As it has been shown by the Mirai botnet attack in 2016, IoT technologies can be used to create record breaking botnets and disrupt more traditional Internet services\cite{203628}. If proper measures are not taken it is likely that more large scale IoT network exploitation and malicious activities will take place. As the manufacturers are pushed by market growth to develop new products faster in order to keep up with their competitors proper security testing is becoming luxury that usually can not be afforded. Moreover, it appears that consumers are not concerned about their device security\cite{iotm}. As studies have shown even Internet connected vehicles can be hacked into and controlled remotely\cite{8071577}. As Internet connected devices are exposed to many more threats than a stand-alone electronic devices steps must be taken to ensure IoT gadgets and Internet connected devices are robust and safe to use.

Due to vast differences in deployment environment as well as technologies in use, proper IoT testing is difficult to ensure. One reason that is severely complicating the testing process is the manufacturers tendency to use third-party technologies and services during the development stages. A single IoT device firmware may be written by a mix of contracted developers known as Original Design Manufacturers (ODM) and in-house developers hired by hardware manufacturer - Original Equipment Manufacturer (OEM), then the application code itself may be supplied by a completely different company\cite{cookbook}. The problem occurs when OEM and ODM reaches the stage where they have to merge their code bases. The ODM may provide only the binary files or an SDK for the OEM. Therefore, if that happens Original Equipment Manufacturer (OEM) which is responsible for distributing firmware, managing it and releasing updates, does not have full access to the code. Moreover, IoT networks are comprised of many different kinds of devices that are responsible for divers functions. Each individual device may be manufactured by different supplier and their OEMs accordingly. As no individual link of the supply chain posses access to the full infrastructure source codes, it may be impossible to thoroughly test the system as a whole. It becomes apparent that black box type penetration testing and vulnerability scans may be the only acceptable strategy in order to secure complicated IoT systems.

As the field in question is rather new, diverse and rapidly developing it is complicated to find knowledgeable specialists and tools developed specifically for IoT penetration testing. There are numerous blogs and projects offering various level of detail guides to IoT penetration testing\cite{github}. Some organizations including IEEE and ETSI have released technology-specific standards as well as security guidelines but non have tried to cover IoT in general\cite{Zhao:2013:SIT:2584913.2585964}. 

The purpose of this project is to try to simplify IoT penetration testing. It summarizes the general concepts of IoT cybersecurity assessment and presents them in a user friendly manner. Moreover, this project tries to address the lack of dedicated IoT penetration testing technologies and provides an extendable framework which purpose is to reuse existing penetration tools to test IoT systems. The tool is designed to require little technical knowledge and is highly modifiable in order to adapt to particular IoT infrastructure.

This report is divided into sections starting topic background research which includes the most commonly used IoT technologies, protocols and communication methods which then shifts to cover IoT penetration testing in general, threat modeling solutions and industry established patterns. Afterwards, project requirements, goals and possible alternative approaches are discussed. Then the report considers numerous design and development decisions that were taken during the course of project progression. Application architecture and implementation details are then described in the following sections. Which are followed by the listing and explanation of penetration tools included in the toolset. Subsequent sections cover application testing, overall project management assessment, final deliverable evaluation and possible future work improving the framework.